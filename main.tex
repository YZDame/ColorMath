\documentclass{ColorMath}
%\excludecomment{analysis}
%\excludecomment{proof}
%\excludecomment{solution}
%\excludecomment{note}
\begin{document}
\maketitle
\tableofcontents

\chapter{ColorMath模板介绍}

\section{模板概述}

ColorMath是一个专为高中数学讲义设计的\LaTeX{}模板。该模板基于ctexbook文档类,针对中文数学教学内容进行了优化,提供了丰富的数学环境和命令,帮助教师制作高质量的数学讲义。

\section{模板特色}

本模板具有以下特色功能:

\begin{itemize}
    \item 完整的中文数学环境支持
    \item 丰富的定理、例题、练习等环境
    \item 自定义的数学符号和命令
    \item 优化的页面布局和字体设置
    \item 支持彩色超链接和书签
\end{itemize}

\section{基本环境示例}

\subsection{定理环境}

\begin{definition}
    设$f(x)$是定义在区间$I$上的函数,如果对于$I$内的任意两点$x_1, x_2$,当$x_1 < x_2$时,都有$f(x_1) < f(x_2)$,则称$f(x)$在区间$I$上是\textbf{严格递增}的。
\end{definition}

\begin{theorem}
    设函数$f(x)$在区间$[a,b]$上连续,在$(a,b)$内可导,如果$f'(x) > 0$在$(a,b)$内恒成立,则$f(x)$在$[a,b]$上严格递增。
\end{theorem}

\begin{proof}
    由拉格朗日中值定理,对于$[a,b]$内任意两点$x_1 < x_2$,存在$\xi \in (x_1, x_2)$,使得
    $$f(x_2) - f(x_1) = f'(\xi)(x_2 - x_1)$$
    由于$f'(\xi) > 0$且$x_2 - x_1 > 0$,所以$f(x_2) - f(x_1) > 0$,即$f(x_2) > f(x_1)$。
\end{proof}

\subsection{例题环境}

\begin{example}
    求函数$f(x) = x^3 - 3x^2 + 2$的单调区间。
\end{example}

\begin{solution}
    首先求导数:$f'(x) = 3x^2 - 6x = 3x(x-2)$

    令$f'(x) = 0$,得$x = 0$或$x = 2$。

    列表分析:
    \begin{center}
        \begin{tabular}{|c|c|c|c|}
            \hline
            $x$     & $(-\infty, 0)$ & $(0, 2)$ & $(2, +\infty)$ \\
            \hline
            $f'(x)$ & $+$            & $-$      & $+$            \\
            \hline
            $f(x)$  & 递增             & 递减       & 递增             \\
            \hline
        \end{tabular}
    \end{center}

    因此,$f(x)$的单调递增区间为$(-\infty, 0)$和$(2, +\infty)$,单调递减区间为$(0, 2)$。
\end{solution}

\subsection{练习环境}

\begin{exercise}
    已知函数$g(x) = 2x^3 - 6x^2 + 6x - 1$,求:
    \begin{enumerate}
        \item $g(x)$的导数
        \item $g(x)$的单调区间
        \item $g(x)$的极值
    \end{enumerate}
\end{exercise}

\section{自定义数学符号}

本模板提供了一些实用的自定义数学符号:

\begin{itemize}
    \item 平行且等于:$a \pxqdy b$(使用命令\verb|\pxqdy|)
    \item 平行四边形:$\pxsbx ABCD$(使用命令\verb|\pxsbx|)
    \item 弧段:$\arc{AB}$(使用命令\verb|\arc{AB}|)
    \item 显示样式求和:$\dsum_{i=1}^n i$(使用命令\verb|\dsum|)
    \item 显示样式乘积:$\dprod_{i=1}^n i$(使用命令\verb|\dprod|)
\end{itemize}


\section{问题与分析}

\begin{question}
    如何判断函数$f(x) = \frac{x^2 - 1}{x^2 + 1}$的奇偶性?
\end{question}

\begin{analysis}
    要判断函数的奇偶性,需要:
    \begin{enumerate}
        \item 确定函数的定义域是否关于原点对称
        \item 计算$f(-x)$
        \item 比较$f(-x)$与$f(x)$的关系
    \end{enumerate}
\end{analysis}

\begin{note}
    在使用本模板时,可以通过注释掉文档开头的\verb|\excludecomment|命令来控制哪些环境在最终文档中显示。

    例如,如果要隐藏所有的解答过程,可以取消注释\verb|\excludecomment{solution}|。
\end{note}

\chapter{ColorMath系列介绍}
\section{RedMath}
结合文献打造高质量教学设计。

\section{CyanMath}
主要以初中竞赛小蓝本为主要参考资料,用于初中竞赛班讲义的制作。

\section{BlueMath}
专注数学竞赛一试讲义,参考书目包括:
\begin{itemize}
    \item 人教A版教材
    \item 全品高考复习手册
    \item 高中小蓝本
\end{itemize}

\section{PurpleMath}
二试题集,参考《数学竞赛研究教程》。

\end{document}